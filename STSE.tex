\def\year{2020}\relax
%File: formatting-instruction.tex
\documentclass[letterpaper]{article} % DO NOT CHANGE THIS
\usepackage{aaai20}  % DO NOT CHANGE THIS
\usepackage{times}  % DO NOT CHANGE THIS
\usepackage{helvet} % DO NOT CHANGE THIS
\usepackage{courier}  % DO NOT CHANGE THIS
\usepackage[hyphens]{url}  % DO NOT CHANGE THIS
\usepackage{graphicx} % DO NOT CHANGE THIS
\urlstyle{rm} % DO NOT CHANGE THIS
\def\UrlFont{\rm}  % DO NOT CHANGE THIS
\usepackage{graphicx}  % DO NOT CHANGE THIS
\frenchspacing  % DO NOT CHANGE THIS
\setlength{\pdfpagewidth}{8.5in}  % DO NOT CHANGE THIS
\setlength{\pdfpageheight}{11in}  % DO NOT CHANGE THIS
\usepackage{bm}
\usepackage{amsfonts}
\usepackage{amsmath}
\usepackage{multirow}
%\nocopyright
%PDF Info Is REQUIRED.
% For /Author, add all authors within the parentheses, separated by commas. No accents or commands.
% For /Title, add Title in Mixed Case. No accents or commands. Retain the parentheses.
 \pdfinfo{
/Title (Spatio-Temporal Structure Enhancement Network)
/Author (Anomymous sumbission)
} %Leave this	
% /Title ()
% Put your actual complete title (no codes, scripts, shortcuts, or LaTeX commands) within the parentheses in mixed case
% Leave the space between \Title and the beginning parenthesis alone
% /Author ()
% Put your actual complete list of authors (no codes, scripts, shortcuts, or LaTeX commands) within the parentheses in mixed case. 
% Each author should be only by a comma. If the name contains accents, remove them. If there are any LaTeX commands, 
% remove them. 

% DISALLOWED PACKAGES
% \usepackage{authblk} -- This package is specifically forbidden
% \usepackage{balance} -- This package is specifically forbidden
% \usepackage{caption} -- This package is specifically forbidden
% \usepackage{color (if used in text)
% \usepackage{CJK} -- This package is specifically forbidden
% \usepackage{float} -- This package is specifically forbidden
% \usepackage{flushend} -- This package is specifically forbidden
% \usepackage{fontenc} -- This package is specifically forbidden
% \usepackage{fullpage} -- This package is specifically forbidden
% \usepackage{geometry} -- This package is specifically forbidden
% \usepackage{grffile} -- This package is specifically forbidden
% \usepackage{hyperref} -- This package is specifically forbidden
% \usepackage{navigator} -- This package is specifically forbidden
% (or any other package that embeds links such as navigator or hyperref)
% \indentfirst} -- This package is specifically forbidden
% \layout} -- This package is specifically forbidden
% \multicol} -- This package is specifically forbidden
% \nameref} -- This package is specifically forbidden
% \natbib} -- This package is specifically forbidden -- use the following workaround:
% \usepackage{savetrees} -- This package is specifically forbidden
% \usepackage{setspace} -- This package is specifically forbidden
% \usepackage{stfloats} -- This package is specifically forbidden
% \usepackage{tabu} -- This package is specifically forbidden
% \usepackage{titlesec} -- This package is specifically forbidden
% \usepackage{tocbibind} -- This package is specifically forbidden
% \usepackage{ulem} -- This package is specifically forbidden
% \usepackage{wrapfig} -- This package is specifically forbidden
% DISALLOWED COMMANDS
% \nocopyright -- Your paper will not be published if you use this command
% \addtolength -- This command may not be used
% \balance -- This command may not be used
% \baselinestretch -- Your paper will not be published if you use this command
% \clearpage -- No page breaks of any kind may be used for the final version of your paper
% \columnsep -- This command may not be used
% \newpage -- No page breaks of any kind may be used for the final version of your paper
% \pagebreak -- No page breaks of any kind may be used for the final version of your paperr
% \pagestyle -- This command may not be used
% \tiny -- This is not an acceptable font size.
% \vspace{- -- No negative value may be used in proximity of a caption, figure, table, section, subsection, subsubsection, or reference
% \vskip{- -- No negative value may be used to alter spacing above or below a caption, figure, table, section, subsection, subsubsection, or reference

\setcounter{secnumdepth}{0} %May be changed to 1 or 2 if section numbers are desired.

% The file aaai20.sty is the style file for AAAI Press 
% proceedings, working notes, and technical reports.
%
\setlength\titlebox{2.5in} % If your paper contains an overfull \vbox too high warning at the beginning of the document, use this
% command to correct it. You may not alter the value below 2.5 in
\title{Temporal-Spatial Consistency Aware Video Inpainting }
%Your title must be in mixed case, not sentence case. 
% That means all verbs (including short verbs like be, is, using,and go), 
% nouns, adverbs, adjectives should be capitalized, including both words in hyphenated terms, while
% articles, conjunctions, and prepositions are lower case unless they
% directly follow a colon or long dash
\author{Anonymous submission}
 \begin{document}

\maketitle

\begin{abstract}
	Video inpainting targets to fill the missing regions in videos, which requires high spatio-temporal consistency.
	However, recent deep-learning based methods focus on temporally smoothening inpainted frames via motion information, which lack an effective mechanism to explore detailed structure clues.
	In this paper, we propose a novel Spatio-Temporal Structure Enhancement Network (STSENet) for video inpainting, which simultaneously considers temporal continuity and detailed structure enhancement. 
	Specifically, STSENet first synthesizes missing optical flow and detailed edge across frames in a parallel way, which are associated with a flow-edge consistency constraint for mutual improvements.
	Then, both edge-preserved flow and temporal coherent edge are used to promote video frame generation, via an structure enhancement mechanism and a flow-guided propagation.
	Consequently, the inpainted frames by STSENet are both temporally stabilized and visually pleasing in detailed structures, with low time cost. 
	Experiments on YouTubeVOS and DAVIS show that STSENet obtains state-of-the-art performance and few artificial flickers, which demonstrate the significance of simultaneous temporal smoothening and structure enhancement in video inpainting.
	
\end{abstract}




\section{Introduction}
\noindent Video inpainting targets to recover the missing contents given videos with spatio-temporal pixels, which can assist lots of practical applications, e.g., video restoration and augmented reality. Compared with image inpainting, video inpainting is much more challenging due to extra time dimension, which requires not only reasonable spatial structures but also stable temporal consistency. Specifically, it is suboptimal to directly apply 2D image inpainting algorithms, such as \cite{yu2018free,Xiong_2019_CVPR}, to individual frames, which will lead to serious artificial effect, such as flickers and jitters. 

To exploit complementary neighbor information across frames, traditional patch-based methods \cite{patwardhan2007video,wexler2004space,newson2014video} recurrently copy the similar patches from unmasked regions and past them to the missing regions. 
This kind of methods depends on a strong hypothesis that the missing content have appeared in neighboring frames, which limits their generalization.
Recently, deep-learning based methods achieve state-of-the-art performance by treating a video as volume and developing CNNs, e.g., 3D convolution operation \cite{wang2019video}, to predict missing content with smooth motion, which is learned from training data.
Among these methods, optical flow is commonly used \cite{Xu_2019_CVPR,Kim_2019_CVPR,Kim_2019_CVPR1} to aggregate context information from neighboring frames, rendering the inpainted content to be temporally consistent.
However, the auxiliary motion compensation brought by optical flow lacks detailed structural clues, i.m., the edge and shape in optical flow are usually blurry, leading to deficiently spatial structure rationality.
Thus, only optical flow is deficient to facilitate video inpainting, in terms of both temporal content smoothing and detailed structure rationality.




\begin{figure}[t]
	\centering
	\includegraphics[width=1.0\columnwidth]{zong} % Reduce the figure size so that it is slightly narrower than the column. Don't use precise values for figure width.This setup will avoid overfull boxes. 
	\caption{The overall pipeline of STSENet. To recover the missing holes in videos, we simultaneously complete edge details and optical flow across frames with a flow-edge consistency constraint for mutual improvements. Then both edge and optical flow are utilized to aid frame inpainting, via an edge-attention mechanism and a flow-guided propagation. Thus, STSENet can produce inpainting results with abundant spatial details and stable temporal consistency.}
	\label{zong}
\end{figure}


To address above issues, we present a novel video inpainting network, called Spatio-Temporal Structure Enhancement Network (STSENet), by simultaneously exploring optical flow and structural clues, to eliminate temporal flickers and enhance spatial detail. 
Specifically, STSENet consists of three modules, as shown in Fig.~\ref{zong}.
Given frames with missing pixels, STSENet first predicts the missing edge and optical flow, according to learned structure knowledge and motion tendency from the training data, via two auxiliary inpainting modules.
These two modules are associated with each other, instead of separate training, via a flow-edge consistency loss.
This results in both edge-enhanced optical flow and temporally smooth edge for mutual improvements, which is superior to single flow generation \cite{Xu_2019_CVPR,Kim_2019_CVPR1}.
Then, both completed edge and optical flow are exploited to guide a Spatio-Temporal Inpainting (STI) module for detail-preserved and temporal coherent video generation.
Specifically, an structure enhancement mechanism is developed for STI to encode detailed structure clues from the completed edge into the generated frames.
For motion guidance in STI, the optical flow is used by propagating complementary pixels from neighboring frames to current frame to alleviate artificial flickers and jitters.


\begin{figure*}[t]
	\centering
	\includegraphics[width=2.0\columnwidth]{sti} % Reduce the figure size so that it is slightly narrower than the column. Don't use precise values for figure width.This setup will avoid overfull boxes. 
	\caption{. }
	\label{sti}
\end{figure*}


Overall, the video inpainting by STSENet is structural reasonable and temporal coherent by exploring auxiliary optical flows and structure edges, which obtains state-of-the-art performance on YouTubeVOS and DAVIS dataset. Our contributions can be summarized as follows.
\begin{itemize}
	\item We propose a novel Spatio-Temporal Structure Enhancement Network (STSENet) for video inpainting, which can simultaneously eliminate temporal flickers and enhance spatial structure details.
	\item We prove that the auxiliary optical flow and structure edges are complementary in video inpainting by using a flow-edge consistency constraint to associate them for mutual improvements.
	\item An structure enhancement mechanism and a flow-guided propagation are developed to utilize respective structural edges and optical flow to promote video inpainting.			
\end{itemize}




\section{Introduction}
\noindent Video inpainting targets to recover the missing contents of given videos, which can assist lots of practical applications, e.g., video restoration and augmented reality. Compared with image inpainting, video inpainting is much more challenging due to extra time dimension. It requires not only reasonable spatial structures but also stable temporal consistency. Specifically, directly applying 2D image inpainting algorithm \cite{yu2018free,Xiong_2019_CVPR} to individual frames is a sub-optimal choice, due to serious artificial effect, flickers and jitters. 

To exploit complementary neighbor information across frames, traditional patch-based methods \cite{patwardhan2007video,wexler2004space,newson2014video} recurrently copy the similar patches from unmasked regions and past them to the missing regions. 
\begin{figure}[t]
	\centering
	\includegraphics[width=1.0\columnwidth]{zong} % Reduce the figure size so that it is slightly narrower than the column. Don't use precise values for figure width.This setup will avoid overfull boxes. 
	\caption{The overall pipeline of STSENet. The ENet and FNet first complete the missing edge and optical flow across frames. Then, under the guidance of auxiliary edge and optical flow, STI can produce structure-preserved and temporally consistent inpainting frame.}
	\label{zong}
\end{figure}
This kind of methods depends on a strong hypothesis that the missing content have appeared in neighboring frames, which limits their generalization.
Recently, deep-learning based methods achieve state-of-the-art performance by treating a video as volume.
They utilize the CNNs, e.g., 3D convolution operation \cite{wang2019video}, to predict missing content with smooth motion, which is learned from training data.
Among these methods, optical flow is commonly used to temporally smooth contents \cite{Xu_2019_CVPR,Kim_2019_CVPR,Kim_2019_CVPR1} by aggregating contextual information.
However, the auxiliary motion compensation brought by optical flow lacks detailed structural clues.
Specifically, the inner textures of objects in optical flow are usually missed, leading to deficient structure rationality.
Thus, it is difficult to obtain both temporal consistent and structure-preserved inpainting video, by only using optical flow.





\begin{figure*}[t]
	\centering
	\includegraphics[width=2.0\columnwidth]{sti} % Reduce the figure size so that it is slightly narrower than the column. Don't use precise values for figure width.This setup will avoid overfull boxes. 
	\caption{The detailed architecture of STSENet. ENet adopts an encoder-decoder architectures with both 3D and 2D convolutions. FNet uses the Resnet101 as backbone, followed by a decoder. STI inpaints the frames in a coarse-to-fine manner. In FNet, $T$ is the number of input frames, and channel $2$ denotes the motion along $x$ and $y$ axis. To save space, all the input masks are omitted.}
	\label{fig:stiNet}
\end{figure*}
To address above issues, we present a novel video inpainting network, called Spatio-Temporal Structure Enhancement Network (STSENet), by simultaneously exploring optical flow and structural clues, to eliminate temporal flickers and enhance spatial detail. 
As shown in Fig.~\ref{zong}, STSENet consists of three modules, which are the ENet, FNet, and STI, respectively.
Given frames with missing pixels, ENet and FNet first predict the missing edge and optical flow, which indicate the structure knowledge and motion tendency.
%according to learned structure knowledge and motion tendency from the training data,
Instead of separate training, these two modules are associated with each other via a flow-edge consistency loss.
This results in both edge-enhanced optical flow and temporally smooth edge.
Then, under the guidance of completed edge and optical flow, STI can generate detail-preserved and temporal coherent inpainting frames.
Specifically, a structure enhancement mechanism is developed to extract and refine the structural clues in the completed edge and encode them into the STI.
For motion guidance, the optical flow is used by propagating complementary pixels from neighboring frames to current frame to alleviate artificial flickers and jitters.

Overall, the video inpainting by STSENet is structural reasonable and temporal coherent with auxiliary optical flows and structure edges.
Our contributions can be summarized as follows.
\begin{itemize}
	\item We propose a novel Spatio-Temporal Structure Enhancement Network (STSENet) for video inpainting, by simultaneously exploring optical flow and structural clues to eliminate temporal flickers and enhance structure detail. 
	\item A flow-edge consistency loss is developed to associate the optical flow and structure edges, which can boost each other.
	\item To explore and refine the latent structure clues in completed edges, a structure enhancement mechanism is specifically designed, which can promote the video inpainting.	
	\item STSENet obtains state-of-the-art performance on YouTubeVOS and DAVIS dataset, with real-time inference speed.	
\end{itemize}





\section{Related Work}




\section{Approach}
\label{sec:approach}

The target of our method is to recover the missing contents in a corrupted video with fine-details and temporal consistence.
%
Our intuition is that there exists complementary information in neighboring frames, which can benefit the inpainting process of each individual frame.
Therefore, in each inference batch, to generate a complete frame \(Y_t\) at time $t$, total $T$ frames $\boldsymbol{X}$ ($T=5$), indexed by $\msset{X}$, are fed to SOVI, as well as corresponding masks $\boldsymbol{M}$ that indicate the missing regions.
%Our method infers each target frame $Y_t$ individually but with hints from its neighboring source frames.
%
To enhance the structural details in the inpainted region, our method consists of three parts: an edge inpainting network ENet that recovers missing edges of the input frames (b) a flow inpainting network FNet that predicts the motion flows in the missing regions, and (c) a spatio-temporal inpainting network STINet that generates plausible video contents.%looks spatially and temporally consistent. 
	%While this forward pass takes multiple frames as input and only completes a single frame, 
%in the inpainted edge maps and videos in a joint learning manner. }
%As shown in Fig.~\ref{fig:stiNet}, our SOVI mainly consists of three parts: (a) an edge inpainting network that recovers structural details of the input frames, (b) a flow inpainting network that completes the dense motions in the missing regions between neighbor frames and the target frame, and (c) a spatio-temporal inpainting network that generates plausible contents that looks spatially and temporally consistent.
The detailed implementation of each part will be explained in the following sections.
%to% enforce the temporal consistency
 



\subsection{Edge Inpainting Network}
\label{sec:edgenet}

Video inpainting suffers from the lack of structural details.
To inpaint the missing regions with fine details, their corresponding structural edges are predicted, which are beneficial to the following frame inpainting process.% reasonable clues
%The edge completion module aims to generate the completed edge maps $\boldsymbol{E}$ for input frames $\boldsymbol{X}$. 
%

Given the input frames $\boldsymbol{X}$, a canny edge detector is first used to extract edge maps $\boldsymbol{E}^{i}$. Notably, the edges of $\boldsymbol{E}^{i}$ in the masked regions are missed. 
%Given the incompleted grayscale images $\boldsymbol{X}^{g}$ of input frame, a canny edge detector is first used to generate initial edge maps . 
Then, the edge inpainting network (ENet) completes the missing edges.
The input of our ENet consists of the grayscale frames $\boldsymbol{X}^{g}$, initial edge maps $\boldsymbol{E}^{i}$, and their corresponding masks $\boldsymbol{M}$.
%
The detailed architecture of ENet is shown in Fig.~\ref{fig:stiNet}, which consists of a generator $G^E$ and a discriminator $D^E$.
$G^E$ contains a 2-layer 3D encoder, eight 2D residual blocks, and a 2-layer 3D decoder. 
The 3D encoder and decoder are designed to learn the spatial-temporal correlation, while the 2D residual blocks are used to enrich the spatial features in a larger receptive field. The discriminator $D^E$ follows the $70\times 70$ PatchGAN architecture \cite{Isola_2017_CVPR}. 
Finally, the inpainted edge maps are obtained by:
\begin{equation}
\label{eq:edgenet}
\boldsymbol{E}=G^E(\boldsymbol{E}^{i},\boldsymbol{X}^{g},\boldsymbol{M}),
\end{equation}

The ENet is trained by playing a minimax game to optimize the generator $G^E$ and the discriminator $D^E$ as
\begin{equation}
\label{eq:loss_e}
\min\limits_{G^E} \max \limits_{D^E} \big(\mathcal{L}^E_{adv}+\lambda_1 * \mathcal{L}^E_{fm}\big),
\end{equation}
where $\mathcal{L}^E_{adv}$ and $\mathcal{L}^E_{fm}$ are respectively the adversarial loss and feature matching loss. 
$\lambda_1$ is a hyper-parameter to balance the two terms.
%
Following the adversarial learning manner, $\mathcal{L}^E_{adv}$ can facilitate ENet to produce plausible edge maps, which is defined by
\begin{equation} \label{eq:edge_adver}
\mathcal{L}^E_{adv}  =\mathbb{E}\big[logD^E(\boldsymbol{E}^{gt},\boldsymbol{X}^{g})\big] +\mathbb{E} \big[log\big(1-D^E ( \boldsymbol{E},\boldsymbol{X}^{g})\big)\big],
\end{equation}
where $E^{gt}$ represents the ground truth edge maps. $\mathcal{L}^E_{fm}$ evaluates the feature-level similarity between ground truth edge maps and predicted edge maps, which helps to create structurally rational edge maps. 
Feature matching loss has been widely used in recent GANs \cite{wang2018high}.
The feature matching loss is defined by:
\begin{equation}
\label{eq:edge_fm}
\mathcal{L}^E_{fm}=\sum_{k=1}^L{\frac{1}{N_k}\left\| D^E_k(\boldsymbol{E}^{gt},\boldsymbol{X}^{g})- D^E_k(\boldsymbol{E},\boldsymbol{X}^{g})\right\|_1},
\end{equation}
where $D^E_k$ is the output of the $k$-th layer in $D^E$, while $N_k$ is the element number of $D^E_k$. 
Note that the discriminator $D^E$ is not optimized by the feature matching loss term. It plays as a feature extractor to optimize the generator $G^E$ for producing plausible edge maps $\boldsymbol{E}$.



\subsection{Flow Inpainting Network}

It is vital to maintain temporal consistency in video inpainting.
To this end, we design a flow inpainting network (FNet) to predict the motion tendency among frames.
%
Similar to ENet, an initial flow maps \(\boldsymbol{O}^i\) for each group of neighboring frames $\boldsymbol{X}$ are first generated using an flow extraction network, such as FlowNet2.0~\cite{Flownet_2017_CVPR}.
Notably, \(\boldsymbol{O}^i\) consists of four flow maps \((O^i_{t\Rightarrow t-7}, O^i_{t\Rightarrow t-3}, O^i_{t\Rightarrow t+3}, O^i_{t\Rightarrow t+7})\) for the input 5 frames.
Then, the proposed FNet $G^F$ is used to complete \(\boldsymbol{O}^i\) by:
\begin{equation}
\label{eq:flownet}
\boldsymbol{O}=G^F(\boldsymbol{O}^{i},\boldsymbol{M}),
\end{equation}
where $\boldsymbol{O}$ denotes the predicted optical flows.
%
The detailed architecture of FNet is shown in Fig.~\ref{fig:stiNet}.

To train the flow inpainting network FNet, the loss function is given by:
\begin{equation}
\label{eq:flow_all}
\mathcal{L}_{flow}=\mathcal{L}^F_{rec}+ \mathcal{L}^F_{har},
\end{equation}
where $\mathcal{L}^F_{rec}$ and $\mathcal{L}^F_{har}$ are respectively $l_1$-reconstruction loss and hard mining loss, following the definition in \cite{Xu_2019_CVPR}. 
Specifically, $\mathcal{L}^F_{har}$ encourages the network to focus on those hard samples in order to avoid blurry texture.

%
%The $l_1$-reconstruction loss is defined as:
%\begin{equation}
%\label{eq:flow_l1}
%\mathcal{L}^F_{rec}=\frac{1}{\left\|\boldsymbol{M} \right\|_1}\left\|(\boldsymbol{O}-\boldsymbol{O}^{gt})\odot \boldsymbol{M}\right\|_1,
%\end{equation}
%where $\boldsymbol{O}^{gt}$ is the ground truth. The symbol $\odot$ denotes the pixel-wise multiplication. Specifically, $\mathcal{L}^F_{rec}$ measures the difference between $\boldsymbol{O}^{gt}$ and $\boldsymbol{O}$.
%The hard mining loss is defined as:
%\begin{equation}
%\label{eq:flow_hard}
%\mathcal{L}^F_{har}=\frac{1}{\left\|\boldsymbol{M}_H \right\|_1}\left\|(\boldsymbol{O}-\boldsymbol{O}^{gt})\odot \boldsymbol{M}_H\right\|_1,
%\end{equation}
%where $\boldsymbol{M}_H$ is the binary mask of hard example regions.
%The hard mining loss encourages the network to focus on those hard samples, which can avoid blurred texture. 

\subsection{Flow-Edge Consistency Loss}

Instead of separately training the two subnetworks ENet and FNet, we train the two networks jointly, because the temporal correlation and structural details can boost each other.
%, thereby rendering them to be mutually improved.
%Thus we will obtain detail-aware optical flow and temporal consistent edge maps.
To achieve this goal, a flow-edge consistency loss is defined as:
%
\begin{equation}
	\label{eq:flow_edge}
	\mathcal{L}_{fec}=\sum_{k}\frac{1}{\left\|M_{t} \right\|_1}\left\|(E_{t}-\phi(O_{t\Rightarrow t-k},E_{t-k}))\odot M_{t}\right\|_1,
\end{equation}
where $\phi(O_{t\Rightarrow t-k},E_{t-k})$ is the warping operation which warps the edge map $E_{t-k}$ to the target frame according to the generated optical flow $O_{t\Rightarrow t-k}$.
$k$ denotes the index of neighboring frames ($k\in \left\{-7,-3,+3,+7 \right\}$). 
%Specifically, $\mathcal{L}_{fec}$ warps the edge maps from neighboring frames to the target frame to penalize the reconstruction loss.
In terms of ENet, $\mathcal{L}_{fec}$ encourages the predicted edge maps to be temporally smoothing, which should conform to the motion tendency in the flow maps $\boldsymbol{O}$. 
In terms of FNet, $\mathcal{L}_{fec}$ constrains the model to focus on the edges in $\boldsymbol{E}$ by emphasizing edge motion in Eq.~\ref{eq:flow_edge}. 
%Consequently, besides of the loss terms $\mathcal{L}_{edge}$ and $\mathcal{L}_{flow}$ for ENet and FNet respectively, $\mathcal{L}_{fec}$ further enhances them to obtain temporally consistent edges and edge-clear flows.
Thus, the total loss function for joint training of ENet and FNet is:
\begin{equation}
	\label{eq:flow}
	\mathcal{L}_{joint}=\mathcal{L}_{edge}+\mathcal{L}_{flow}+ \mathcal{L}_{fec}.
\end{equation}







\subsection{Spatio-Temporal Inpainting Network}

Combining the inpainted edge maps $\boldsymbol{E}$ and flow maps $\boldsymbol{O}$, a spatio-temporal inpainting network (STINet) is designed to obtain the final target frame $Y_t$.
%\cxj{I modified to STINet to make it consistent with ENet and FNet. If it is ok, pls modify Fig 2 (STI Module) accordingly.}
%under the guidance of the edges and motion tendency, which is helpful to the completion process of the target frame...

%\cxj{Maybe move this paragraph backward. }




Our STINet adopts a coarse-to-fine architecture, which consists of a coarse network and a refinement network, as shown in Fig.~\ref{fig:stiNet}.
%
The input of STINet is the concatenation of $\boldsymbol{X}$, $\boldsymbol{E}$, and $\boldsymbol{M}$.
First, the coarse network consists of a set of 3D convolutions to capture the temporal information, which targets to generate initial rough completed frames $\boldsymbol{Y}^i$ with textures. 
The 3D coarse network is able to incorporate information from neighboring frames by convolutions of the time dimension.
Second, the refinement network is implemented with 2D convolutional layers to enhance structural details and refine the initial completed frame to produce a visually realistic frame with fine textures.

\begin{figure}[t]
	\centering
	\includegraphics[width=0.65\columnwidth]{SEM} % Reduce the figure size so that it is slightly narrower than the column. Don't use precise values for figure width.This setup will avoid overfull boxes. 
	\caption{The architecture of SEM.}
	\label{SEM}
\end{figure}

To fully exploit the structural information in $\boldsymbol{E}$, a structure attention module (SAM) is designed for STINet.
The core insight of SAM is to capture the spatial correlation between structural edges and video contents, which is easier to embed into STINet than original edges.
As shown in Fig.~\ref{SEM}, two separate convolution blocks are first applied to the predicted edge maps $\boldsymbol{E}$.
Then, the extracted high-level video features and the refined structural features are interacted to calculate the latent structure-texture correlation via matrix multiplication. 
%\cxj{Explain reshape?}
%
Next, a SoftMax operation is used to obtain a normalized attention map.
%
Besides the short-range edge information in $\boldsymbol{E}$, the normalized attention map contains extra long-range correlation between structure and high-level features of video content.
Finally, the normalized attention map is applied to the intermediate video features, and the structure information is thus embedded in the STINet.
After introducing structure information, the inpainted video content by STINet becomes more structurally and semantically realistic.



%\cxj{Introduce the loss for the generated $Y_t$ first. Then temporal consistence loss guided by the flows. }
Inspired by \cite{nazeri2019edgeconnect}, a reconstruction loss $\mathcal{L}^{I}_{rec}$ is used to measure the difference between predicted video frame and the ground truth video frame $\widetilde{Y}_t$:
%\mdf{First  between the predicted video frame and the ground truth video frame $\widetilde{Y}_t$.}
% \mdf{measures the difference between .. }. It consists of three terms:
\begin{equation}
	\mathcal{L}^{I}_{rec}=\mathcal{L}^{I}_{l1}+\lambda_3 *\mathcal{L}^{I}_{per}+\lambda_4 *\mathcal{L}^{I}_{sty},
\end{equation}
%
where the first term is pixel $l_1$ loss to evaluate generation quality. Different from \cite{nazeri2019edgeconnect}, we penalize both the coarse and fine predictions, which is defined by:
\begin{equation}
	\begin{aligned}
		\mathcal{L}^{I}_{rec}&=\frac{1}{\left\|\boldsymbol{M} \right\|_1}\left\|(\boldsymbol{Y}^i-\widetilde{\boldsymbol{Y}})\odot \boldsymbol{M}\right\|_1\\ &+\lambda_5*\frac{1}{\left\|M_t \right\|_1}\left\|(Y_t-\widetilde{Y}_t)\odot M_t\right\|_1.
	\end{aligned}
\end{equation}
%
The second term is the perceptual loss \cite{gatys2015neural}, which compares the semantic contents between the generated frame and ground truth. The third term is the style loss, designed to prevent "checkerboard" artifacts \cite{Sajjadi_2017_ICCV} caused by transpose convolution.


%\cxj{Explain why we need the adv loss. Are we the first to use Ladv or we follow existing method?}
Besides, we introduce an extra adversarial loss $\mathcal{L}^I_{adv}$ to promote the visual reality of the generated frame:
%$\mathcal{L}^I_{adv}$ is defined as:
\begin{equation}
	\label{eq:inp_adver}
	\mathcal{L}^I_{adv}=\min\limits_{G^I} \max \limits_{D^I}\mathbb{E}[logD^I(\widetilde{Y}_t)]+\mathbb{E}log[1-D^I(Y_{t})],
\end{equation}
where $G^I$ is the STI and $D^I$ possesses the same architecture as that of $D^E$, except the input dimension.% $\mathcal{L}^I_{adv}$ enforces the generated frame to be more realistic.




%\cxj{I would suggest introduce the forward pass in a subsection and the joint learning with flows in another subsection. }

To smooth temporal flickers, the motion tendency information in $\boldsymbol{O}$ is utilized via a flow warping constraint and a temporal ensemble module.
Specifically, a flow warping loss is proposed by warping the neighboring frames into the target frame:
\begin{equation}
	\label{eq:inp_flow}
	\mathcal{L}^I_{flo}=\sum_{\widehat{t}\in\mathcal{T}}\left\| Y_t-\phi(O_{t\Rightarrow \widehat{t}},Y_{\widehat{t}}) \right\|_1,
\end{equation}
where $\mathcal{T}=\{t-7,t-3\}$. $\phi(O_{t\Rightarrow \widehat{t}},Y_{\widehat{t}})$ warps $Y_{\widehat{t}}$ to $Y_{t}$ using flow $O_{t\Rightarrow \widehat{t}}$.
Therefore, $\mathcal{L}^I_{flo}$ expects that all the neighboring frames can be well warped to the target frame with small reconstruction loss, rendering the inpainted frames to be temporal consistent.
The temporal ensemble module is designed to aggregate the temporal neighboring frames, of which the architecture is shown in Fig.~\ref{fig:stiNet}.
%\(O_{t\Rightarrow t-3}\) is used to warp the inpainted $Y_{t-3}$ to aid the current $Y_{t}$, which provides a high temporal coherence.


%, which will generate a temporal smooth fine-detailed $Y_{t}$.
Finally, the loss function for STINet is:
\begin{equation}
	\label{eq:inpain_all}
	\mathcal{L}_{sti}=\mathcal{L}^{I}_{rec}+\lambda_2 * \mathcal{L}^I_{adv}+ \mathcal{L}^I_{flo},
\end{equation}
where $\mathcal{L}^{I}_{rec}$, $\mathcal{L}^I_{adv}$, and $\mathcal{L}^I_{flo}$ are respectively reconstruction loss, adversarial loss and flow warping loss.

Notably, the inpainted edges and flows provide structural details and temporal correlation, respectively. 

















%$\boldsymbol{\psi}_j$ denotes the activation maps of $j_{th}$ layer in a network. $N_j$ is the number of elements in layer $\boldsymbol{\psi}_j$. We use layers $relu_{1\_1}$, $relu_{2\_1}$, $relu_{3\_1}$, $relu_{4\_1}$ and $relu_{5\_1}$ of the VGG-19 network pre-trained on the ImageNet dataset for this loss.
%The third term is the style loss, designed to prevent "checkerboard" artifacts \cite{Sajjadi_2017_ICCV} caused by transpose convolution. $G^{\boldsymbol{\psi}_j}$ is the Gram matrix calculated by auto-correlation of $\boldsymbol{\psi}_j$. The layers are the same as that of perceptual loss.








\section{Experiments}

In order to evaluate the effects of different components in SOVI and compare it with state-of-the-art approaches, we conduct series of experiments on two datasets, YouTubeVOS \cite{xu2018Youtube} and DAVIS \cite{davis_2017}.
%We test on two datasets, YouTubeVOS \cite{xu2018youtube} and DAVIS \cite{davis_2017}, to compare the proposed STSENet with state-of-the-art methods. %Several ablation studies are conducted to prove the effectiveness of spatial details and temporal information in video inpainting.
\subsection{Experimental Settings}
\textbf{Dataset.} 
YouTubeVOS and DAVIS are widely used for evaluation in recent video inpainting approaches.
YouTubeVOS is a large-scale dataset that contains 4,453 YouTube video clips. These videos are close to real-world scenarios
 with 70+ common objects. 
The videos are officially split into three parts, 3,471 for training, 474 for validating, and 508 for testing. 
DAVIS dataset is for video object segmentation containing 150 video clips, among which 60 randomly sampled clips are for testing of object removal. And the rest part is used for training.
The videos are complex with occlusions, fast motion, and various objects. 

\noindent \textbf{Mask Setting.} Considering various real-world applications, we test our method on four kinds of mask settings in this paper. 
They are different in shapes and positions of the missing regions.
\begin{enumerate}
	\item Fixed square mask. The size and position of the missing square region are fixed through the whole video. 
	\item Moving square mask. The position and size of the square mask change over frames. 
	\item Free-from mask. We apply irregular mask which imitates hand-drawn masks on each frame, following \cite{liu2018partialinpainting}. 
	\item Foreground object mask. This type of masks is defined to line out the foreground objects in videos, which is used for object removal.
\end{enumerate}


\noindent \textbf{Implementation Details.} 
In the data preparation stage, we randomly sample a clip every 40 frames from each video in the datasets.
The video frames are resized into $256\times256$.
%
The training process consists of three steps.
First, we  train the ENet and FNet jointly using Adam optimizer with $\beta=(0.9, 0.999)$.
The learning rate is set to $1e-4$ for $N^E$ and $G^F$ and $1e-5$ for $D^E$. Then, the STINet is trained with the learning rate of $1e-4$ for $G^I$, and $4e-4$ for $D^I$. Finally, the temporal ensemble module is trained with the learning rate of $1e-4$.  We do not use weight decay in training.
As for the hyper-parameters, $\lambda_1=10.0,\lambda_2=5.0, \lambda_3=0.1$.

\noindent \textbf{Evaluation Metrics.} 
We use three commonly-used metrics to quantitatively evaluate the performance of our method. They are structural similarity index (SSIM) \cite{wang2004image}, peak signal-to-noise ratio (PSNR), and Fr{\'e}chet Inception Distance (FID) \cite{heusel2017gans}. 
Besides, the quantitative metrics can not be used in the experiments of foreground object removal, since there is no ground truth available. So we conduct a user study for video foreground object removal. 
%\cxj{Where these three metrics are used compared with object removal?}
%
%Besides, since there is no ground truth for the experiments of object removal, 



\begin{figure*}[t]
	\centering
	\includegraphics[scale=0.127]{viszong} % Reduce the figure size so that it is slightly narrower than the column. Don't use precise values for figure width.This setup will avoid overfull boxes. 
	\caption{Visualization for video inpainting on YouTubeVOS. Our method can produce frames with finer sturcture than existing methods. }
	\label{viszong}
\end{figure*}




\subsection{Ablation Study}
To demonstrate the effectiveness of proposed components in our method, we first conduct ablation study on YouTubeVOS. 

\subsubsection{Effect of Structure Clues in STINet.}



To evaluate the effects of structural information in video inpainting, we successively add different parts to the baseline and compare their performances. Three variants are used: 'STI', '+edge input', and '+SAM'. 
'STI' is the baseline using only spatio-temporal inpainting network, without either structure clues or temporal information. '+edge input' is the model that uses the predicted edge maps as input. '+SAM' is the model to which structure attention module is applied.
\begin{figure}[!ht]
	\centering
	\includegraphics[width=0.97\columnwidth]{edgevis} % Reduce the figure size so that it is slightly narrower than the column. Don't use precise values for figure width.This setup will avoid overfull boxes. 
	\caption{Visualized effects of exploring structure edges in video inpainting. It's obvious that we can obtain more detail-clear results with structure guidance.}
	\label{edgevis}
\end{figure}
The quantitative results are shown in Table~\ref{tab:sem}. 
%As shown, the network achieves better results when edge information is utilized, comparing '+edge input' to 'STI'. It indicates that edge clues are effective guidance in video inpainting, which helps the network to predict more accurate frames.
We can see that '+edge input' brings large improvement over the baseline STI.
It indicates that the edge clues are effective guidance in video inpainting, which helps the network to predict accurate missing content.
When we further add SAM to STI, extra improvement is obtained.
It demonstrates that, compared with original edge, the latent structure information,~\emph{i.m.}~spatial correlation between structural edge and video texture, can be better embedded and absorbed by STI.
The above analyses prove that the edge clues are effective guidance in video inpainting, which helps the network to predict more accurate frames.

Furthermore, the visualized results are given in Fig.~\ref{edgevis}. It's obvious that after introducing the edge structure, the inpainted frames become visually better, with sharper object contours. Besides, the edge maps predicted by our method are reasonable and clear, which well represents the structure information and show the strong edge inpainting ability of ENet.
Thus, it is crucial to explore structural details when inpainting the videos.






\begin{figure}[t]
	\centering
	\includegraphics[width=1.0\columnwidth]{flowvis} % Reduce the figure size so that it is slightly narrower than the column. Don't use precise values for figure width.This setup will avoid overfull boxes. 
	\caption{Visualized optical flow predicted by our method.}
	\label{flowvis}
\end{figure}
\begin{figure}[t]
	\centering
	\includegraphics[width=1.0\columnwidth]{flow_vis} % Reduce the figure size so that it is slightly narrower than the column. Don't use precise values for figure width.This setup will avoid overfull boxes. 
	\caption{Visualized effects of temporal information.}
	\label{flow_vis}
\end{figure}

\subsubsection{Effect of Temporal Information in STINet.}
Temporal consistency is also an important factor in video inpainting. In our method, we utilize temporal information from the inpainted flows to smoothen the artificial flickers via the developed flow-guided warping and temporal ensemble module. 
%We use '+flow' to denotes the model 
First of all, we provide the visualized results in Fig.~\ref{flowvis}, where the proposed FNet can effectively complete the missing flows with fine boundaries.
Then, the quantitative results are given in Table~\ref{tab:sem}.
'Ours' is denoted as our model that utilizes flow-guided warping and temporal ensemble module.
It can be seen that the result is further improved after utilizing temporal information. It proves that the complementary contents can be aggregated by motion, which is helpful in video inpainting. 
As shown in Fig.~\ref{flow_vis}, the motion information can smoothen artificial flickers, resulting in temporal consistent inpainting results, and the color changes between neighboring frames become less obvious after employing flow.
Furthermore, it should be noted that the performance improvement from temporal information is smaller than that from structure information in this paper.
The reason maybe that our base model has certain ability to utilize complementary information of neighboring frames.
%a flow warping loss to smoothen the artificial flickers and propagate complementary information from neighboring frames. 
%It is achieved by $\mathcal{L}_{fec}$, which demonstrates that structure information can also boost the completion of optical flow.
%To evaluate the impact of temporal smoothening in STI, we compare two baselines, STI and STI w/o flow.
%As shown in Table~\ref{tab:edge}, STI works better than STI w/o flow, values...!!!.
%The visualized results are shown in Fig.~\ref{flow_vis}.
%It can be observed that the artificial flickers are alleviated when motion guidance is involved. 
Both the quantitative and qualitative results prove that the motion information is beneficial to temporal consistency as well as inpaitning.
%Besides, he visulization results in Fig.~\ref{flowvis} shows that $\mathcal{L}_{fec}$ encourages the network to predict optical flow  which demonstrates that structure information can also boost the completion of optical flow.

%We only use the first three mask settings on YouTubeVOS.
%\cxj{for what reason? page limit of the paper? How about others? Can we result in the same conclusion for different mask settings?}


%We discussed four variants of our method. We 
%\begin{enumerate}
%	\item STI: The Spatio-Temporal Inpainting network without guidance of either edge or flow.
%	\item +edge:
%	\item Free-from mask. We apply irregular mask which imitates hand-drawn masks on each frame, following \cite{liu2018partialinpainting}. 
%	\item Foreground object mask. This type of masks are defined to line out the foreground objects in videos, which is used for object removal.
%\end{enumerate}
%\noindent \textbf{Baselines.} Several variants of STSENet are defined as following. (1) STI w/o flow: The Spatio-Temporal Inpainting network without flow guidance \cxj{with or without EdgeNet and SEM?}. (2) STI w/o edge: The Spatio-Temporal Inpainting network without structure guidance. (3) STSENet w/o $\mathcal{L}_{fec}$: The Spatio-Temporal Inpainting network with guidance of both structure and motion, but $\mathcal{L}_{fec}$ is not used. (4) STSENet is the model which uses all modules proposed in this paper. 
%
%\cxj{We discussed four variants of our method. Use that version. }
\begin{figure}[!h]
	\centering
	\includegraphics[width=1.0\columnwidth]{vis_forg} % Reduce the figure size so that it is slightly narrower than the column. Don't use precise values for figure width.This setup will avoid overfull boxes. 
	\caption{Visualized object foreground removal. The red mask in input frames indicates the object that we want to remove.}
	\label{vis_forg}
\end{figure}
\subsection{Comparisons with Existing Methods.}
We compare our results with state-of-the-art inpainting methods \cite{nazeri2019edgeconnect,wang2019video,Kim_2019_CVPR1,Xu_2019_CVPR}. 
As shown in Table~\ref{tab:sem}, Our structure-oriented method achieves better results than other methods, which demonstrates the effectiveness of introducing structural clues into video inpainting.
Specifically, the additional time cost brought by edge enhancement is negligible.
When the flow inpainting module is removed in SOVI, the inference speed is almost twice faster, and our performance is still competitive.
This indicates that structure clues can bring strong promotion to video inpainting, 
We also give qualitative results in Fig.~\ref{viszong}. Compared with existing methods, inpainting results predicted by our method are more realistic with finer details. We can observe that the frames predicted by our method contain more sharper object contours. This is achieved by the effectiveness of structure information in video inpainting.









%\noindent \textbf{Effect of Flow-Edge Consistency Loss.}
%Flow-edge consistency loss $\mathcal{L}_{fec}$ is designed for mutual improvement of optical flow and edge maps.
%To demonstrate the effectiveness of $\mathcal{L}_{fec}$ in training, we compare the performances between STSENet w/o $\mathcal{L}_{fec}$ and STSENet. We use standard end-point-error (EPE) metric to evaluate the completion of optical flow. Besides, the well-completed flow and edge aid the final inpainting results, so the quality of final inpainting results also reflects the impact of $\mathcal{L}_{fec}$.
%The quantitative results are shown in Table~\ref{tab:lfec}. It indicates that $\mathcal{L}_{fec}$ plays a positive role in prediction of flow and edge, which is helpful for video inpainting. 
%\begin{table}[t]
%	\caption{The effect of structure clues and temporal smoothening in STSENet. The mask number denotes the indexes of mask setting in the section Experimental Settings. We compare STI,STI w/o SEM, and  in three aspects of metrics.}\smallskip
%	\centering
%	\resizebox{.95\columnwidth}{!}{
%		\smallskip\begin{tabular}{c|c|c|c|c|c|c|c|c|c }
%			\hline
%&\multicolumn{3}{c|}{Fixed Square}& \multicolumn{3}{c|}{Moving Square}&\multicolumn{3}{c}{Free-Form}\\
%\cline{2-10} 
%&PSNR & SSIM & FID & PSNR & SSIM & FID & PSNR & SSIM & FID\\
%\hline
%STI &28.0174 &0.9494  &  42.7164   &	
%33.8131 &  0.9705  &8.2390	& 
%30.0680& 0.9390 & 20.6358
%\\
%\hline
%+edge input  &29.5242 &  0.9520& 36.2097   &	
%37.6630	& 0.9798 &3.5161    &	
%33.8206	&0.9659  &    6.6651  \\
%\hline
%
%+SEM &29.9918 &  0.9533 &  27.4198  &	
%38.2433	& 0.9807 &   2.5083  &	
%35.7783	&0.9712  &   5.8786  \\
%\hline
%
%+flow &\textbf{30.0590} &\textbf{0.9543}&   \textbf{27.2431}  &
%\textbf{38.8186} & \textbf{0.9824} & \textbf{2.3455} &
%\textbf{35.9613}  & \textbf{0.9721}&  \textbf{ 5.8694} \\
%
%\hline
%			
%			
%		\end{tabular}
%	}
%	\label{tab:edge}
%\end{table}






\subsection{User Study on Video Object Removal}
Evaluation metrics can not fully reflect the quality of inpainted videos. So in addition to qualitative comparison, we also conduct a user study on DAVIS dataset to evaluate the visual quality of our method. We compare our method with the strong methods \cite{nazeri2019edgeconnect,wang2019video,Kim_2019_CVPR1,Xu_2019_CVPR}.
In each test, the origianl video, result produced by our method and result of another method are shown at the same time. The participants are allowed to play the videos multiple times to notice the differences of results of different methods.
we invite 30 participants for the user study. Each participant has to watch 20 random videos carefully, and choose which method is visually better. We list the preferred proportion of each methods compared to ours. The results are reported in Table~\ref{tab:userstudy}, which shows that our method is preferred by the users. Visualization in Fig.~\ref{vis_forg} also demonstrates that the results generated by our methods are visually better than existing methods.
\begin{table}[t]
	\caption{The result of user study.}\smallskip
	\tiny
	\centering
	\resizebox{0.7\columnwidth}{!}{
		\smallskip\begin{tabular}{c|c}
			\hline
			Comparing Methods&Preferred Proportion\\
			\hline
			Ours/Nazeri et al.&0.9083/0.0917\\
			\hline
			Ours/Wang et al. &0.8015/0.1985\\
			\hline
			Ours/Kim et al. b &0.7357/0.2643\\
			\hline
			Ours/Xu et al. &0.5941/0.4059\\
			\hline
			
			
		\end{tabular}
	}
	\label{tab:userstudy}
\end{table}






\section{Conclusion}
In this paper, we propose a novel structure-oriented video inpainting method, which utilizes sturcture information to generate fine-detailed frames. We first complete edge maps, which indicate structure details in frames. Then we generate frames under the guidance of structure information. Besides, we synthesize missing optical flow to constrain the temporal consistency of final outputs.
Experiments on YouTubeVOS and DAVIS datasets demonstrate the effectiveness of our method of utilizing structure preserving in video inpainting.


% Specifically, we jointly complete edge maps and optical flow with a consistency loss, which helps to obtain temporal consistent edge and edge-clear optical flow. Then with the guidance of spatial details and motion tendency, spatio-temporal inpainting network is designed to predict the final video with fine spatial details few artificial flickers.
%The state-of-the-art performance on both YouTube-VOS and DAVIS prove the effectiveness of exploring structural clues and optical flow in video inpainting.





 

\bibliographystyle{aaai} \bibliography{aaai2}
\end{document}
